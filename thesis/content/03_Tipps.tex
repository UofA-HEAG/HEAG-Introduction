\chapter{Tipps}


\section{References}
\label{sec:references}

The use of \texttt{autoref} from the \texttt{hyperref} package is really useful.
It automatically writes the type, e.\,g.\ Figure, section, subsection, of your reference.
Like this, not only the number of your reference, but also the type is used for the link.
Furthermore, no linebreaks between the type and the number will occur.
To use it, just replace \texttt{\textbackslash ref} by \texttt{\textbackslash autoref}.

Here is an example to show the differences between the commands:
\ref{sec:references} vs.\ \autoref{sec:references}.

For the appendix, the use of \texttt{\textbackslash autoref} will create the following:

\autoref{app:magic} for the chapter, and \autoref{sec:selected_data} for the section.
For sections, you might want to have it differently.
Therefore, a specific command has been added in the preamble.
Just use \texttt{\textbackslash aref} instead of \texttt{\textbackslash autoref}, and you will get the following:

\aref{sec:selected_data} for the section.


\section{Citations}

The adavantage of using the citationstyle author year (instead of numbers or abbreviations) is that a reader from your field will most probably know the corresponding publication and will not need to check the bibliography.
Moreover, a name is easier to remember than just a number.
Therefore, you will make it easier for the reader to read your thesis (and you want to make it as comfortable for your referee as possible).

In principle, there are two situations where you want to cite something:
\texttt{\textbackslash cite}: \cite{malaga} suggests to use model XY.
\texttt{\textbackslash citep}: The model describes the formation of stars  \citep{malaga}.


\section{Graphics}

\section{Tables}

\section{Structure of the Thesis}